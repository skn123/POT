\documentclass[twoside,11pt]{article}

% Any additional packages needed should be included after jmlr2e.
% Note that jmlr2e.sty includes epsfig, amssymb, natbib and graphicx,
% and defines many common macros, such as 'proof' and 'example'.
%
% It also sets the bibliographystyle to plainnat; for more information on
% natbib citation styles, see the natbib documentation, a copy of which
% is archived at http://www.jmlr.org/format/natbib.pdf

\usepackage{jmlr2e}
\usepackage[utf8]{inputenc}
% Definitions of handy macros can go here

\newcommand{\dataset}{{\cal D}}
\newcommand{\fracpartial}[2]{\frac{\partial #1}{\partial  #2}}

% Heading arguments are {volume}{year}{pages}{submitted}{published}{author-full-names}

\jmlrheading{1}{2000}{1-48}{4/00}{10/00}{Rémi Flamary and Nicolas Courty}

% Short headings should be running head and authors last names

\ShortHeadings{POT : Python Optimal Transport }{Flamary and Courty}
\firstpageno{1}

\begin{document}

\title{POT Python Optimal Transport}

\author{\name Rémi Flamary \email remi.flamary@unice.fr \\
       \addr Lagrange Laboratory\\
       Université Côte d'Azur\\
       Nice, 06 108, France
       \AND
       \name Nicolas Courty \email courty@univ-ubs.fr \\
       \addr Somewhere over the rainbow\\
       University of Kansas\\
       Definitely not in Kansas anymore}

\editor{}

\maketitle

\begin{abstract}%   <- trailing '%' for backward compatibility of .sty file
Optimal transport has been reintroduced recently to the machine
learning community thanks to novel efficient optimization procedures that
allow for medium to large scale applications. We propose an open source
toolbox that contain several implementation of the key optimal
transport ideas for the community. The toolbox contain not only
re-implementation from a number of founding works of OT for machine
learning but also provides generic solvers that can be used for more
fundamental research.
\end{abstract}

\begin{keywords}
  Optimal transport, divergence, optimization, domain adaptation
\end{keywords}

\section{Introduction}



% Acknowledgements should go at the end, before appendices and references

\acks{We would like to acknowledge ANR Project OATMIL. }

% Manual newpage inserted to improve layout of sample file - not
% needed in general before appendices/bibliography.

\newpage

% \appendix
% \section*{Appendix A.}
% \label{app:theorem}

% % Note: in this sample, the section number is hard-coded in. Following
% % proper LaTeX conventions, it should properly be coded as a reference:

% %In this appendix we prove the following theorem from
% %Section~\ref{sec:textree-generalization}:




\vskip 0.2in
\bibliography{biblio}

\end{document}
